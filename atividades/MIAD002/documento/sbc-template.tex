\documentclass[12pt]{article}

\usepackage{sbc-template}

\usepackage{graphicx,url}

\usepackage[brazil]{babel}   
%\usepackage[latin1]{inputenc}  
\usepackage[utf8]{inputenc}  

\sloppy

\title{Introdução à analise de dados\\ Atividade dois}

\author{Felipe Carvalho\inst{1}, Felipe Menino\inst{2}}

\address{FATEC "Jessen Vidal" 
    \email{\{felipe.carvalho69, felipe.carlos\}@fatec.sp.gov.br}
% \nextinstitute
%   Department of Computer Science -- University of Durham\\
%   Durham, U.K.
% \nextinstitute
%   Departamento de Sistemas e Computação\\
%   Universidade Regional de Blumenal (FURB) -- Blumenau, SC -- Brazil
%   \email{\{nedel,flavio\}@inf.ufrgs.br, R.Bordini@durham.ac.uk,
%   jomi@inf.furb.br}
}

\begin{document} 

\maketitle

% \begin{abstract}
%   This meta-paper describes the style to be used in articles and short papers
%   for SBC conferences. For papers in English, you should add just an abstract
%   while for the papers in Portuguese, we also ask for an abstract in
%   Portuguese (``resumo''). In both cases, abstracts should not have more than
%   10 lines and must be in the first page of the paper.
% \end{abstract}
     
\begin{resumo} 
    A aplicação dos conceitos vistos no mini-curso de introdução a análise de dados é extremamente importante para a fixação dos conteúdos. Esta é uma lista de atividades para auxiliar o aluno na fixação de todo o conteúdo.
\end{resumo}

\section{Contexto}

A Universidade Federal do Oeste da Bahia (UFOB), disponibiliza dados nominais de alunos que trancaram suas matrículas na universidade. Os dados disponibilizados são referentes aos anos de 2014 à 2017.

Ao verificar estes dados, algumas dúvidas surgiram, como por exemplo, qual o curso tem mais registros, ou qual o ano.

Os são disponibilizados aqui: http://dados.gov.br/dataset/estudantes-com-matricula-trancada

Veja que, neste caso, não sabemos se todos os dados estão completamente atualizados, vamos apenas supor que eles estão completos.

\section{Atividade}

Como demonstrado ao longo das atividades do curso, à analise de dados é basicamente uma atividade utilizada para responder perguntas através de diferentes métodos de tratamento de manipulação dos dados.

Com base nisto, faça a análise do conjunto de dados \textbf{MIAD002}, que está no repositório do minicurso, dentro do diretório \textbf{atividades}.

Com este conjunto de dados, responda as seguintes questões:

\begin{itemize}
    \item Qual o curso com maior quantidade de matrículas trancadas no ano de 2015 ?
    \item Faça um \textbf{plot} com a variação entre os anos do número de matrículas trancadas por curso.
    \item Qual semestre (de todos os anos) contém a maior taxa de matrículas trancadas ?
\end{itemize}

% \section{Referências}

% Bibliographic references must be unambiguous and uniform.  We recommend giving
% the author names references in brackets, e.g. \cite{knuth:84},
% \cite{boulic:91}, and \cite{smith:99}.

% The references must be listed using 12 point font size, with 6 points of space
% before each reference. The first line of each reference should not be
% indented, while the subsequent should be indented by 0.5 cm.

% \bibliographystyle{sbc}
% \bibliography{sbc-template}

\end{document}
