\documentclass[12pt]{article}

\usepackage{sbc-template}

\usepackage{hyperref}
\usepackage{graphicx, url}

\usepackage[brazil]{babel}   
\usepackage[utf8]{inputenc}  

\sloppy

\title{Introdução à análise de dados\\ Atividade dois}

\author{Felipe Carvalho\inst{1}, Felipe Menino\inst{1}}

\address{Instituto Nacional de Pesquisas Espaciais 
	\email{felipe.carvalho@inpe.br},
	\email{felipe.carlos@fatec.sp.gov.br}
}

\begin{document} 

\maketitle
     
\begin{resumo} 
    A aplicação dos conceitos vistos no minicurso de introdução à análise de dados é extremamente importante para a fixação dos conceitos. Desta forma, esta lista de exercícios busca criar situações que ajudem o aluno na aplicação e fixação de tais conceitos.
\end{resumo}

\section{Contexto}

A Universidade Federal do Oeste da Bahia (UFOB), disponibiliza dados nominais de alunos que trancaram suas matrículas na universidade. Os dados disponibilizados referentes aos anos de 2014 à 2017 estão no \href{http://dados.gov.br/dataset/estudantes-com-matricula-trancada}{Portal Brasileiro de Dados Abertos}

\section{Atividade}

Como apresentado ao longo das atividades do curso, a analise de dados é basicamente uma atividade utilizada para responder perguntas através de diferentes métodos de tratamento e manipulação dos dados.

Com base nisto, faça a análise do conjunto de dados \textbf{MIAD002}, que está no repositório do minicurso, dentro do diretório \textbf{atividades}.

Com este conjunto de dados, responda as seguintes questões:

\begin{itemize}
    \item Qual o curso com maior quantidade de matrículas trancadas no ano de 2015 ?
    \item Faça um \textbf{plot} com a variação entre os anos do número de matrículas trancadas por curso.
    \item Qual semestre (de todos os anos) contém a maior taxa de matrículas trancadas ?
\end{itemize}

\end{document}
