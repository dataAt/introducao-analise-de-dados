\documentclass[12pt]{article}

\usepackage{sbc-template}

\usepackage{graphicx,url}

\usepackage[brazil]{babel}   
%\usepackage[latin1]{inputenc}  
\usepackage[utf8]{inputenc}  

\sloppy

\title{Introdução à análise de dados\\ Atividade um}

\author{Felipe Carvalho\inst{1}, Felipe Menino\inst{1}}

\address{FATEC "Jessen Vidal" 
    \email{felipe.carvalho@inpe.br},
    \email{felipe.carlos@fatec.sp.gov.br}
% \nextinstitute
%   Department of Computer Science -- University of Durham\\
%   Durham, U.K.
% \nextinstitute
%   Departamento de Sistemas e Computação\\
%   Universidade Regional de Blumenal (FURB) -- Blumenau, SC -- Brazil
%   \email{\{nedel,flavio\}@inf.ufrgs.br, R.Bordini@durham.ac.uk,
%   jomi@inf.furb.br}
}

\begin{document} 

\maketitle

% \begin{abstract}
%   This meta-paper describes the style to be used in articles and short papers
%   for SBC conferences. For papers in English, you should add just an abstract
%   while for the papers in Portuguese, we also ask for an abstract in
%   Portuguese (``resumo''). In both cases, abstracts should not have more than
%   10 lines and must be in the first page of the paper.
% \end{abstract}
     
\begin{resumo} 
    A aplicação dos conceitos vistos no mini-curso de introdução a análise de dados é extremamente importante para a fixação dos conteúdos. Esta é uma lista de atividades para auxiliar o aluno na fixação de todo o conteúdo.
\end{resumo}

\section{Contexto}

Esta seção apresenta um pequeno contexto dos dados que serão analisados na atividade.

Titulo do conjunto de dados: Morbidade - Acidentes de Transporte Terrestre Segundo Tipo de Vítima
Disponível em: http://sage.saude.gov.br

Este conjunto de dados, representa uma serie temporal, do ano de 2001 até 2015, com dados relacionados ao número de mortes no trânsito de acordo com cada tipo de vítima.

\section{Atividade}

Como demonstrado ao longo das atividades do curso, à analise de dados é basicamente uma atividade utilizada para responder perguntas através de diferentes métodos de tratamento de manipulação dos dados.

Com base nisto, faça a análise do conjunto de dados \textbf{MIAD001}, que está no repositório do minicurso, dentro do diretório \textbf{atividades}.

Com este conjunto de dados, responda as seguintes perguntas:

\begin{itemize}
    \item Qual é o tipo mais comum em acidentes ?
    \item Qual ano apresentou a maior taxa geral de mortalidade por acidentes ?
    \item Qual ano apresentou a maior taxa de mortalidade em automóveis ?
    \item Qual o desvio padrão entre todos os anos, em acidentes sofridos por Motociclistas ?
    	\begin{itemize}
    		\item Lembre-se: Em probabilidade, o desvio padrão ou desvio padrão populacional é uma medida de dispersão em torno da média populacional de uma variável aleatória. Wikipédia
    	\end{itemize}
\end{itemize}

% \section{Referências}

% Bibliographic references must be unambiguous and uniform.  We recommend giving
% the author names references in brackets, e.g. \cite{knuth:84},
% \cite{boulic:91}, and \cite{smith:99}.

% The references must be listed using 12 point font size, with 6 points of space
% before each reference. The first line of each reference should not be
% indented, while the subsequent should be indented by 0.5 cm.

% \bibliographystyle{sbc}
% \bibliography{sbc-template}

\end{document}
