\documentclass[12pt]{article}

\usepackage{sbc-template}

\usepackage{hyperref}
\usepackage{graphicx,url}

\usepackage[brazil]{babel}   
%\usepackage[latin1]{inputenc}  
\usepackage[utf8]{inputenc}  

\sloppy

\title{Introdução à análise de dados\\ Atividade um}

\author{Felipe Carvalho\inst{1}, Felipe Menino\inst{1}}

\address{Instituto Nacional de Pesquisas Espaciais 
    \email{felipe.carvalho@inpe.br},
    \email{felipe.carlos@fatec.sp.gov.br}
}

\begin{document} 

\maketitle
     
\begin{resumo} 
    A aplicação dos conceitos vistos no minicurso de introdução à análise de dados é extremamente importante para a fixação dos conceitos. Desta forma, esta lista de exercícios busca criar situações que ajudem o aluno na aplicação e fixação de tais conceitos.
\end{resumo}

\section{Contexto}

Este conjunto de dados, representa uma serie temporal, do ano de 2001 até 2015, com dados relacionados ao número de mortes por transportes terrestres. Os dados são disponibilizados pela \href{http://sage.saude.gov.br}{Sala de Apoio à Gestão Estratégica}.

Fazer a análise deste conjunto de dados, pode nos ajudar a entender os problemas de morbidade relacionados a transportes terrestres.

\section{Atividade}

Como apresentado ao longo das atividades do curso, a analise de dados é basicamente uma atividade utilizada para responder perguntas através de diferentes métodos de tratamento e manipulação dos dados.

Com base nisto, faça a análise do conjunto de dados \textbf{MIAD001}, que está no repositório do minicurso, dentro do diretório \textbf{atividades}.

Com este conjunto de dados, responda as seguintes perguntas:

\begin{itemize}
    \item Qual é o tipo mais comum em acidentes ?
    \item Qual ano apresentou a maior taxa geral de mortalidade por acidentes ?
    \item Qual ano apresentou a maior taxa de mortalidade em automóveis ?
    \item Qual o desvio padrão entre todos os anos, em acidentes sofridos por Motociclistas ?
    	\begin{itemize}
    		\item Em probabilidade, o desvio padrão ou desvio padrão populacional é uma medida de dispersão em torno da média populacional de uma variável aleatória. Wikipédia
    	\end{itemize}
\end{itemize}

\end{document}
