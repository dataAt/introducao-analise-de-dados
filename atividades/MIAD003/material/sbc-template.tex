\documentclass[12pt]{article}

\usepackage{sbc-template}

\usepackage{hyperref}
\usepackage{graphicx, url}

\usepackage[brazil]{babel}   
\usepackage[utf8]{inputenc}  

\sloppy

\title{Introdução à análise de dados\\ Atividade três}

\author{Felipe Carvalho\inst{1}, Felipe Menino\inst{1}}

\address{Instituto Nacional de Pesquisas Espaciais
    \email{felipe.carvalho@inpe.br},
    \email{felipe.carlos@fatec.sp.gov.br}
}

\begin{document} 

\maketitle
     
\begin{resumo} 
    A aplicação dos conceitos vistos no minicurso de introdução à análise de dados é extremamente importante para a fixação dos conceitos. Desta forma, esta lista de exercícios busca criar situações que ajudem o aluno na aplicação e fixação de tais conceitos.
\end{resumo}

\section{Contexto}

Atualmente na \textbf{Google Play Store} existem milhares de aplicativos, com as mais variadas características. Um conjunto de dados disponibilizados no \href{https://www.kaggle.com/}{Kaggle} traz informações gerais sobre 10 mil aplicativos disponibilizados na loja.

O interessante deste conjunto é tentar fazer verificações dos tipos mais populares de aplicativos, para quem sabe, ao começar um novo aplicativo, você já saber qual tendência seguir.

Para esta análise, um conjunto amostral de 1000 registros será utilizado, porém, encorajamos você a depois que finalizar a atividade, acessar a \href{https://www.kaggle.com/lava18/google-play-store-apps}{página dos dados} e fazer a análise com todos os dados.

\section{Atividade}

Como apresentado ao longo das atividades do curso, a analise de dados é basicamente uma atividade utilizada para responder perguntas através de diferentes métodos de tratamento e manipulação dos dados.

Com base nisto, faça a análise do conjunto de dados \textbf{MIAD003}, que está no repositório do minicurso, dentro do diretório \textbf{atividades}.

Com este conjunto de dados, responda as seguintes perguntas:

\begin{itemize}
    \item Qual a categoria de aplicativo mais comum na \textbf{Google Play Store} ?
    \item Qual a categoria dos aplicativos com maior média de avaliações ?
    \item A maior quantidade de \textit{reviews} pertence a qual aplicativo ?
    \item Há mais aplicativos pagos ou grátis neste conjunto de dados ?
\end{itemize}

\end{document}
